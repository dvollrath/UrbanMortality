\documentclass[10pt]{article}
\input{/Users/dietz/Dropbox/Tools/article.tex}

\begin{document}
\onehalfspacing 

\subsection{For the text}
The location-specific elasticities are central to our quantitative assessment of the UMT. In Table 4, we show variations in how those elasticities are determined, and that the importance of the UMT remains robust. Row 1 shows our baseline results, with the rural elasticity set to $\epsilon_r = 1.2$, and the formal and informal elasticities calibrated so that the model matches the targets shown in Table 2. Those targets are the average urbanization rate and informal share in our baseline sample of 43 countries.

We can instead calibrate the model for \textit{each} country separately, given a rural elasticity of $\epsilon_r = 1.2$, and get values of $\epsilon_f$ and $\epsilon_l$ unique to each country. Rows 2 and 3 of Table 4 show several pieces of information. First, while the mean value of the formal elasticity is high compared to our baseline elasticity, the median formal elasticity from the 43 separate calibrations is only slightly higher than the baseline elasticity. For the informal elasticity, both the mean value and the median from the 43 separate calibrations are very close to our baseline value. Both indicate that using averages from those 43 countries as our targets did not skew our results. Columns 5 and 6 of rows 2 and 3 show that the results if we simulate the model using the mean or median elasticities. In both cases, the difference in urbanization and informal share without the UMT is similar in size to our baseline calibration, as can be seen by comparing rows 2 and 3 to row 1. 

In the next panel of the table, we look instead at whether the chosen initial value for $\epsilon_r$ is responsible for our results. The original value of $\epsilon_r = 1.2$ was drawn from Lee (XXXX), but within that source there is some variation in the size of the implied rural elasticity. In Row 4 if we set $\epsilon_r = 1.6$, at the high end, and re-calibrate the model, the implied value of the formal elasticity if $\epsilon_f = 1.500$, with an informal elasticity of $\epsilon_l = 0.813$. All the elasticities are shifted up in size, but note that the pattern across elasticities remains similar, with the informal elasticity around half of the rural value, and the formal elasticity near the rural one. In this case, the model says the UMT explains slightly less urbanization (8.2 percentage points) and informalization (6.8 percentage points) than our baseline model. In Row 5 we instead set $\epsilon_r = 1.0$, at the low end, and re-calibrate once more. As expected, all the elasticities are shifted down, with a formal elasticity of $\epsilon_f = 1.189$ and an informal elasticity of $\epsilon_l = 0.553$. The results imply that the UMT explains 10.2 percentage points of urbanization, and 9.8 percentage points of informalization, both slightly more than our baseline. These results show that what is important in driving our outcomes is not the absolute level of the elasticities, but their pattern across locations. 

This holds true in Row 6, where we approach setting the rural elasticity from an entirely different direction. Here, rather than starting with an aggregate estimate of the elasticity, we build up the aggregate elasticity from underlying sources. In the Appendix, we have written out an explicit model, but the aggregate elasticity is made up of four parts: (a) the elasticity of wages with respect to the number of workers, which can be inferred from the factor share of land and capital, (b) agglomeration effects, measured as the elasticity of productivity with respect to the number of workers, (c) housing effects, measured as the elasticity of housing prices with respect to population, weighted by the expenditure share on housing, and (d) amenity effects, measured by the elasticity of amenity value with respect to population, weighted by a measure of how important amenities are in welfare. For each of these four components, we can draw on estimates from the literature to build up an aggregate elasticity for a given location.

For the rural location, we find a factor share of land and capital in developing countries of around 0.56 (cites XXXXXX). For agglomeration, there is no significant evidence of any in rural areas (cites XXXX), so we set this to zero. For housing, the expenditure share appears to be close to 10\% (cites XXXXX) while the elasticity of house prices with respect to population is close to 2.0 (cites XXXXX). Finally, for amenities sources indicate a welfare weight of 0.1 (cites XXXX), but there is little evidence that amenities are sensitive to population size, in particular in developing countries (cites XXXX). In sum, the rural elasticity is thus $\epsilon_r = 0.56 + 0 + .1\times2.0 + 0 = 0.760$. This is lower than the original aggregate elasticity we set, but if one examines the results in Row 6 of Table 4, the implied effects of the UMT are not substantially different if we use this to calibrate the model, and are in fact larger with this lower assumed rural elasticity. 

Also in Row 6 can be found the calibrated values of the formal (1.064) and informal (0.425) elasticities. These were set by the calibration so that we matched our targets exactly, not built up from the underlying sources. However, we can perform similar calculations for these locations, and those elasticities can be found in Row 7. For the formal location, the factor share of capital and land is closer to 0.6 (cites XXXXX), slightly higher than in the rural location. We find estimates of the agglomeration elasticity of 0.036 (cites XXXX). For housing, a similar expenditure share is found of 10\%, and the elasticity of house prices with respect to population is set to 4.0 (cites XXXX, see Appendix). Finally, for amenities, there again does not appear to be any significant relationship of amenity value to population size (cites XXXX), so this is set to zero. In sum, the formal elasticity is $\epsilon_f = 0.6 - 0.036 + 0.1\times 4.0 + 0 = 0.964$, where note the agglomeration effect is subtracted, as $\epsilon_f$ is capturing the negative effect on welfare from additional population, whereas agglomeration has a positive effect. In informal locations, we perform a similar exercise, and find that the factor shares of capital and land are 0.3, while the agglomeration effect is actually larger than in formal areas, at 0.075 (cites XXXX). For housing, the expenditure share is still 10\%, but the elasticity of house prices with respect to population is found to be 2.0 (cites XXXX). Finally, similar to the formal location, there does not appear to be any significant effect on amenities of population size, so this effect is zero. In sum, the informal elasticity is $\epsilon_l = 0.3 - 0.075 + 0.1 \times 2.0 + 0 = 0.425$.

Comparing Row 7, with these inferred elasticities, to Row 6, where we calibrated the formal and informal elasticities, one can see they are quite close.\footnote{One note is that the simulation in Row 7 is not disciplined to match the target data, and overstates the urbanization rate and informal share relative to the observed data in 2005. While it delivers a similar prediction of the effect of the UMT, we do not want to overstate the importance of that similarity.} Certainly in terms of their relative sizes, the calibration in Row 6 does a good job of matching the formal and informal elasticities. All the elasticities in Row 7 are shifted down relative to our baseline in Row 1, but this is not driving our results, as our baseline understates the explanatory power of the UMT compared to Row 7. The elasticities in Row 7 are also drawn in many cases from data on rich countries, and they may understate the negative effects of population growth on welfare in developing countries. We have also not taken any explicit account of the possible negative effects of population size on health itself. For those reasons, we don't believe our baseline elasticities are problematic, despite being larger than those shown in Row 7. 

One thing that would overturn the importance of the UMT is if the true elasticity in informal areas were much larger, so that they did not have this advantage of being able to absorb population. If we make extreme assumptions regarding the informal location, we could assume that $\epsilon_l$ was larger. One obvious area is amenities, where we could assume informal locations have amenities very sensitive to population size. Setting the elasticity of amenities with respect to population to 0.25, we could generate an informal elasticity of $\epsilon_l = 0.3 - 0.075 + 0.1 \times 2.0 + 0.25 = 0.675$. Using this as our informal elasticity, as in Row 8, leads to less explanatory power for the UMT. Even in this case, though, the UMT explains 6.9 percentage points of urbanization, and 4.1 percentage points of the informal share. 
 
\subsection{Individual utility}
Let individual utility be described by the following
\begin{equation}
    V_j = \omega_c \ln c_j + \omega_h \ln h_j + \omega_q \ln Q_j + \omega_d \ln (1/CDR_j).
\end{equation}
where $c_j$ is consumption of non-housing goods and services, and $h_j$ is consumption of housing services. The preference weights $\omega_c$ and $\omega_h$ fulfill, without loss of generality, $\omega_c + \omega_h = 1$. 

The amenity value $Q_j$ and the crude death rate $CDR_j$ are taken as given by an individual in location $j$. They choose, however, the amounts of $c_j$ and $h_j$ to purchase.  The budget constraint facing an individual in location $j$ is
\begin{equation}
    w_j = c_j + p_j h_j
\end{equation}
where $p_j$ is the price of housing (relative to consumption goods), and $w_j$ is the wage. Optimizing over consumption and housing, we get that $c_j = \omega_c w_j$ and $h_j = \omega_h w_j/p_j$. This results in utility of
\begin{equation}
    V_j = \ln w_j - \omega_h \ln p_j + \omega_q \ln Q_j + \omega_d \ln (1/CDR_j) + \Omega,
\end{equation}
where $\Omega = \omega_c \ln \omega_c + \omega_h \ln \omega_h$. 

For our purposes, we care about the growth rate of utility in each location, which is given by
\begin{equation}
    \hat{V}_j = \hat{w}_j - \omega_h \hat{p}_j + \omega_q \hat{Q}_j - \omega_d \hat{CDR}_j.
\end{equation}
We can evaluate each of these separate terms to find the effect of population growth on utility growth.

\textbf{Wages:} To determine the wage in a given location, we assume a production function of the form
\begin{equation}
    Y_j = A_j K_j^{\alpha_j} X_j^{\beta_j} N_j^{1 - \alpha_j - \beta_j}
\end{equation}
where $A_j$ is productivity, $K_j$ is capital, and $X_j$ is land. Note that the shares $\alpha_j$ and $\beta_j$ are unique to a given location. Assuming that labor earns its marginal product, the wage in location $j$ is given by
\begin{equation}
    w_j = {1 - \alpha_j - \beta_j} A_j K_j^{\alpha_j} X_j^{\beta_j} N_j^{- \alpha_j - \beta_j}.
\end{equation}
We can allow explicitly for agglomeration effects by letting productivity be a function of the labor force, as in
\begin{equation}
    A_j = B_j N_j^{\gamma_j}
\end{equation}
where $B_j$ is the inherent productivity in location $j$, and $\gamma_j>0$ captures the agglomeration effect in location $j$. 

With this specification for productivity, and the expression for the wage, the growth rate of the wage is given by
\begin{equation}
    \hat{w}_j = \hat{a}^w_j + (\gamma_j - \alpha_j - \beta_j)\hat{N}_j,
\end{equation}
where $\hat{a}^w_j = \hat{B}_j + \alpha_j \hat{K}_j$ is the growth of productivity and capital that is independent of the growth rate in the labor force. The effect of growth in the labor force may be positive or negative, depending on how big the agglomeration effects ($\gamma_j$) are relative to the importance of capital and land in production ($\alpha_j + \beta_j$).

In terms of our baseline model, we can write the above as
\begin{equation}
    \hat{w}_j = \hat{a}^w_j + \epsilon^w_j \hat{N}_j,
\end{equation}
where $\epsilon^w_j = \gamma_j - \alpha_j - \beta_j$ is the elasticity of the wage with respect to population size. 

\textbf{Housing:} The second term in utility growth involves the price of housing. Given some limitations on housing due to land constraints and/or regulations, then it should be the case that prices are rising with $N_j$. We could write this in a reduced form as
\begin{equation}
    \hat{p}_j = \hat{a}^h_j \epsilon_j^h \hat{N}_j
\end{equation}
where $\epsilon^h_j$ is the elasticity of the housing price with respect to population size. The value $\hat{a}^h_j$ is exogenous growth in the housing price in location $j$ for any reason unrelated to population size. 

\textbf{Amenities:} The third term in utility growth involves amenties. Assuming some effect of population size on the quantity (or implicit value) of amenities available, then we could write, again in reduced form,
\begin{equation}
    \hat{Q}_j = \hat{a}^q_j + \epsilon^q_j \hat{N}_j.
\end{equation}
Here, $\hat{a}^q_j$ is exogenous growth in the value of amenities, and $\epsilon^g_j$ is the elasticity of amenities with respect to population size. 

\textbf{Crude death rate:} As stated in the main paper, we are taking the growth of crude death rates to be exogenous, and unrelated to population size.

\textbf{Reduced form welfare growth:} Combining the information about the different components of utility growth, we can write
\begin{equation}
    \hat{V}_j = G_j - \epsilon_j \hat{N}_j - \omega_d \hat{CDR}_j
\end{equation}
where 
\begin{equation}
    G_j = \hat{a}^w_j - \omega_h \hat{a}^h_j + \omega_q \hat{a}^q_j
\end{equation}
is the exogenous growth in wages, housing prices, and amenities, where each are weighted as in the utility function. Similarly, the combined elasticity term, $\epsilon_j$ is
\begin{eqnarray}
    \epsilon_j &=& \epsilon^w_j - \omega_h \epsilon^h_j + \omega_q \epsilon^q_j \\
               &=& \gamma_j - \alpha_j - \beta_j - \omega_h \epsilon^h_j + \omega_q \epsilon^q_j \label{EQN_epsilon}
\end{eqnarray}
where the second line shows the explicit role of agglomeration effects ($\gamma_j$) and the production function parameters ($\alpha_j$ and $\beta_j$) on the elasticity. WHile the value of $\gamma_j$ is expected to be positive, note that the rest of these terms would act to make $\epsilon_j$ negative. The values of $\alpha_j$ and $\beta_j$ indicate how quickly wages decline with the number of workers. The relationship of housing prices to population size is presumably positive, so the term $\omega_h \epsilon^h_j$ is positive, and so acts to make the overall elasticity negative as well.

\subsection{Parameter values}
The expression in (\ref{EQN_epsilon}) shows us what explicit types of information we need to quantify the aggregate elasticity of welfare with respect to population size. For each in turn, we can evaluate them based on outside literature. Ideally, we would be able to provide a separate estimate of the parameter by location, although as we will describe below, this will not always be feasible.

\textbf{Factor shares:} The values of $\alpha_j$ and $\beta_j$ (which capture the importance of capital and land in production in location $j$, respectively), can be estimated from factor share information. Individually, $\alpha_j$ and $\beta_j$ would be equal to the factor share of capital and land, assuming factor markets are competitive. Alternatively, their sum, $\alpha_j + \beta_j$, could be recovered if we know the \textit{labor} share in a location. 

The major issue here is that factor shares are not reported by location, but typically only by sector. So in order to leverage factor share information, we first have to take some position on what sectors belong to which locations. Of course, many of these sectors operate in all locations (i.e. retail) so there is not a one-to-one mapping. Rather, we think of the locations as having different weights on which sectors are relatively important. 

In the rural location, the agriculture is obviously going to be the dominant sector, with smaller weights on manufacturing and services activities. For the formal sector, we believe that the factor shares will be driven by sectors involving heavy manufacturing (i.e. chemicals or equipment), wholesale trade, information, and services like finance. For the informal sector, the factor shares will depend mainly on sectors like light manufacturing (i.e. textiles), retail and/or commerce, and personal services. 

Valentinyi and Herrendorf (2008) provide estimates of the combined land and capital shares, $\alpha_j + \beta_j$, for several broad sub-sectors of the U.S. economy. In agriculture, this share is 0.54, suggesting that $\alpha_r + \beta_r = 0.54$ may be a decent estimate. Restuccia and Santaeulalia-Llopis (2017) find a similar share of 0.58 for Malawi, and Chen, Restuccia, and Santaeulalia-LLopis (2017) give a share of 0.58 in Ethiopia. For agriculture, we use a value of 0.56. 

For the formal location, Valentinyi and Herrendorf's data is less useful, because it remains (for us) at a very high level of aggregation, with services all lumped together and the aggregate of manufacturing consumption (with a capital/land share of 0.40) also being somewhat broad. Young (2010) looks at labor shares at a more disaggregate level for the United States, and over a longer period of time, but does not make the same adjustments for proprietors income that Valentinyia nd Herrendorf do. With that caveat in mind, Young's numbers show high labor shares, and hence low values of $\alpha_f + \beta_f$ for several sectors we consider associated with formal locations. Utilities, both gas and electric, have implied values of $\alpha_f + \beta_f = 0.657$. Finance has an implied value of $\alpha_f + \beta_f = 0.556$, and communications gives an implied value of 0.503. Heavy industries such as chemicals and oil products have labor shares that gives values around $\alpha_f + \beta_f = 0.50$. Zuleta et al (2009) use data from Colombia, giving us something more appropriate in a developing context. Their capital shares suggest values of $\alpha_f + \beta_f = 0.743$ for public services, $\alpha_f + \beta_f = 0.487$ for manufacturing, and $\alpha_f + \beta_f = 0.625$ for finance. Using KLEMS data from India, which has reported labor shares by sub-sector, we find that heavy manufacturing industries, as well as business services, have implied values of $\alpha_f + \beta_f$ greater than 0.6. Based on all this information, we set the value of $\alpha_f + \beta_f = 0.6$.

In constrast, for sectors we associate with informal locations, Zuleta et al find $\alpha_l + \beta_l = 0.180$ for commerce and food service, $\alpha_l + \beta_l = 0.230$ for social and personal services, and $\alpha_l + \beta_ = 0.274$ for transport and storage. This is in line with the numbers from Young (2010), who find $\alpha_l + \beta_l = 0.227$ for trade in the U.S., and $\alpha_l + \beta_l = 0.311$ for services in general. Certain manufacturing sub-sectors, such as apparel, also have estimates of high labor shares, suggesting $\alpha_l + \beta_l = 0.152$. Valentinyi and Herrendorf's broad service sector has a capital share of 0.34, above these individual estimates. The KLEMS data from India reports the smallest implied values of $\alpha_l+\beta_l$ to be around 0.150-0.400 for personal services, health and social work, and food service. To be conservative, we set the $\alpha_l + \beta_l = 0.3$.

\textbf{Agglomeration effects:} Duranton (2016) reports agglomeration effects of 0.054 for all workers using data from Colombia, and reports that his preferred estimate for the elasticity of wages with respect to population (precisely our $\gamma$ term) is about 0.05. As part of his paper, though, he also reports a separate estimate for formal workers \textit{only}, and that indicates $\gamma_f = 0.036$, lower than the overall elasticity. Given that lower estimate, it follows that the informal agglomeration effect is actually higher than 0.05. Without a precise reported value, we use an assumed value of $\gamma_l = 0.075$, roughly symmetric around the overall value with the formal estimate. Developing nations like China and India have estimated values of around 0.10-0.12 (Combes, D'emurger, and Shi, 2015; Chauvin, Glaeser, and Tobio, 2013), although there is no distinction between formal and informal workers. However, the low value for formal workers reported by Duranton is similar to estimates done for developed countries, with presumably all (or nearly all) formal employment in cities. The higher estimates from China and India would be consistent with larger agglomeration effects in informal locations.

\textbf{Housing prices:} For the effect of prices, we need both the expenditure share, $\omega_h$ and the elasticity, $\epsilon^h_j$. For the expenditure share, the average for housing expenditures for our baseline sample, from XXXXX in 19XX, is about 10\%. This low value is due to their relative poverty. For the same set of countries, the expenditure share on food is over 50\%. However, even if we were to assume a larger expenditure share on housing, similar to shares in developed countries, this will not result in an appreciable different in the results we find.

The elasticity, $\epsilon^h_j$, is the percent change in prices given the percent change in the population, $N_j$. We can leverage several types of results. The elasticity of housing supply with respect to price is found to be quite low in developing nations, between 0.1 and 0.5 (Malpezzi and Mayo, 1997; Lall et al, 2007). Inverting this elasticity to get the elasticity of price with respect to the supply of housing, we get estimates of between 2 and 10. Assuming that population and the number of housing units are proportional within a given location (which does not require that the household size is similar across locations), then this is also the elasticity of housing price with respect to population size.

These elasticities refer to urban housing prices, but do not distinguish between formal and informal locations. To assess the difference in these elasticities, we use the results of Saiz (2010), who finds that the elasticity of prices with respect to the number of households varies by the strictness of land use regulations measured by the Wharton Regulation Index (WRI) of Gyuorko, Saiz, and Summers (2008), as well as by the geographic limitations on housing. The impact of geography alone makes Saiz's estimated elasticity twice as large in the least geographically restricted cities compared to the most restricted, and variation in the WRI only expands that difference. If we use very restricted cities as an analogy for formal locations, and very unrestricted cities as an analogy for informal ones, then this implies that the elasticity should be (at least) twice as large in formal locations as in informal ones. Taking the low end elasticity estimate of an elasticity of 2.0 from above to capture the informal elasticity, $\epsilon^h_l = 2.0$, then this would imply that $\epsilon^h_f = 4.0$, or more. Limiting the difference between these two works against our outcomes, so we will work with this implied gap of 2.0. For rural areas, there are not comparable estimates that we are aware of, and we set $\epsilon^h_r = 2.0$ as well to match informal locations.

Combined with the expenditure share of 10\%, this indicates that $\omega_f \epsilon^h_f = 0.1 \times 4.0 = 0.4$, while $\omega_l \epsilon^h_l = 0.1 \times 2.0 = 0.2$, and $\omega_r \epsilon^h_r = 0.1 \times 2.0 = 0.2$. 

\textbf{Amenities:} 
XXXXX Remi - I did not get to update this section today. Will try to look again tomorrow XXXXXX

For amenities, we need to establish both the utility weight $\omega_q$, as well as their elasticity with respect to population, $\epsilon^q_j$. Estimates indicate that the elasticity is close to zero, after controlling for natural amenities such as weather and coastal locations that are insensitive to population, as in Albouy (2008). Duranton (2016) finds no effect of amenities on wages within cities, indicating that there is no trade-off at work, consistent with a utility weight of $\omega_q = 0$. Chauvin et al (2016) find no relationship of weather and wages in China and India, perhaps indicating that at low levels of development amenities (of whatever form) are not valued very highly in welfare, again indicating a value of $\omega_q = 0$. Albouy (2012) finds that the quality of life does not significantly differ by city size, suggesting $\epsilon^q_j$ is close to zero as well. 

If we did allow for a non-zero weight on amenities, $\omega_q >0$, then Diamond's (2016) results on the difference in the response to amenities by schooling suggests that formal locations likely have a larger elasticity than informal locations, $\epsilon^q_f > \epsilon^q_l$. This is consistent with our calibrated finding of a larger formal elasticity, but the other evidence suggests that these elasticities are both quite small, even if non-zero.

\begin{table}[htb]
\begin{center}
\begin{footnotesize}
\caption{\textbf{DIFFERENT ASSUMPTIONS ON LOCATION ELASTICITIES}} \label{tab_elasticity}
\begin{tabular}{lccccccc}
\midrule
         & \multicolumn{3}{c}{Elasticities:} & \multicolumn{2}{c}{Actual vs. no-UMT} &Welfare & Model  \\ \cmidrule(lr){2-4} \cmidrule(lr){5-6} 
Scenario & Rural ($\epsilon_r$) & Formal ($\epsilon_f$) & Informal ($\epsilon_l$) & Urb. Rate & Inf. Share & Ratio & Urb. Diff \\ \midrule
\input{table_jv_base_elastic.txt}
\midrule
\end{tabular}
\end{footnotesize}
\end{center}
Notes: XXXXXX The bottom section shows outcomes using only parts of our analysis (factor shares, agglomeration, etc.). The final column is just a check for me on how badly we are missing the observed urbanization rate. You can see in lines 11 and 12 that we are way off. That's why I'm wary of talking about them. You cannot take the results with the UMT seriously in those rows. XXXXXXXXX
\end{table}
\end{document}

